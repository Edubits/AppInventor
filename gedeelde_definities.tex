\usepackage[dutch]{babel}

\usepackage{parskip}
\usepackage{graphicx}
\usepackage{hyperref}
\usepackage{needspace}
\sloppy
%%% ---------------  onze definities ----------------

% app inventor kleuren
\definecolor{aiblue}{rgb}{0.65,.78,0.9}
\definecolor{aigreen}{rgb}{0.81,.87,0.6}

% \ai voor 'App Inventor' (ik word zo moe van steeds die hoofdletters in te moeten typen :-) )
\newcommand{\ai}[0]{\emph{App Inventor} \nolinebreak}

% gebruik \block voor een block in de AppInventor-block editor   ( voor nu: italic bold )
\newcommand{\block}[1]{\colorbox{aiblue}{ \texttt{#1} }}

\newcommand{\menuitem}[1]{\colorbox{aigreen}{ #1 }} 

\newcommand{\bestand}[1]{\url{#1}}  % eigenlijk is het gebruik van url hier niet helemaal correct omdat de link nergens naartoe wijst, maar het ziet er wel OK uit. 

% Run op telefoon
\newcommand{\runOpTelefoon}[2][0pt]{
    \marginpar{\captionsetup{type=table} \vspace{#1}
    \begin{minipage}[t][#2]{1.23in}
    \smallskip
    \footnotesize 
    \checkoddpage
    \ifoddpage
	   \RaggedRight 
    \else
	   \RaggedLeft
    \fi
    \includegraphics[width=.615in]{screenshots/run_op_telefoon}
	    
    \end{minipage}
    }
}
