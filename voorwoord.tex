\chapter*{Voorwoord}
\addcontentsline{toc}{chapter}{Voorwoord}

Dit boek is geschreven in het kader van Onderzoek van Onderwijs aan de Eindhoven School of Education, Technische Universiteit Eindhoven. Het boek is bedoeld voor gebruik als introductie tot programmeren in het vak Informatica in de bovenbouw van het voortgezet onderwijs. Aan de hand van enkele applicaties voor Android telefoons zul je de principes van programmeren leren.

Zoals op de vorige pagina al gemeld wordt is dit gepubliceerd onder een Creative Commons Naamsvermelding-license. Dit wil zeggen dat je het materiaal naar eigen inzicht mag veranderen en (de al dan niet veranderde versie) verder mag verspreiden (wel met vermelding van de oorspronkelijke auteurs), zolang je er geen geld voor vraagt. De auteurs (Robin, Fran\c{c}ois en Coen) vinden het prettig iets te horen van mensen die het materiaal op enige manier gebruiken. Heb je een nieuw hoofdstuk? Staat er een fout in? Laat het horen! Vast dank. Dan weten we ook of het de moeite waard is als we er tijd in stoppen om het materiaal up to date te houden. 

Het oorspronkelijke materiaal is gemaakt in \LaTeX. De sources zijn te vinden in 
\emph{GitHub}: \url{https://github.com/Edubits/AppInventor}, maar je kunt de inhoud natuurlijk ook kopi\"eren in je favoriete tekstverwerker om er dan mee aan de slag te gaan. 



\section*{Gebruikte notaties}
\addcontentsline{toc}{section}{Gebruikte notaties}
In dit boek worden enkele notaties gebruikt om het voor jou als lezer makkelijker leesbaar te maken. In deze sectie geven we hier een opsomming van.

\begin{description}
   \item[\block{Block}] Blocks zul je gebruiken om te programmeren. Blocks in \ai worden weergegeven als puzzelstukjes. 
   \item[\menuitem{Knop} of \menuitem{menu}] Menuopties of knoppen worden op deze manier aangegeven.
   \item[\emph{Term}] Belangrijke termen worden schuingedrukt om deze extra te benadrukken.
   \item[\lefthand\ Hints] Naast de tekst worden soms belangrijke hints of tips gegeven.
   \item[Opgaven] Opgaven worden aangeduid met een blauwe balk. 
     \begin{opgave}
       \opgVraag
	Hier staat de vraag.
       \opgUitwerking
         In sommige gevallen wordt ook een uitwerking gegeven.
     \end{opgave}
   \item[Belangrijk] In sommige gevallen willen we je iets belangrijks vertellen, bijvoorbeeld waarschuwing, een tip of een uitleg van een belangrijk begrip. Deze worden voorzien van een rode balk.
     \begin{derivation}{Waarschuwing}
       Hier staat een waarschuwing, lees deze zorgvuldig!
      \end{derivation}
    \item[Uitproberen!] \runOpTelefoon{} Regelmatig kom je de afbeelding hiernaast tegen. Dit betekent dat je de app waar je op dat moment mee bezig bent uit mag proberen op je telefoon of in de emulator.
\end{description}
