\chapter*{Voorwoord}
\addcontentsline{toc}{chapter}{Voorwoord}

Dit boek is geschreven in het kader van Onderzoek van Onderwijs aan de Eindhoven School of Education, Technische Universiteit Eindhoven. Het boek is bedoeld voor gebruik als introductie tot programmeren in het vak Informatica in de bovenbouw van het voortgezet onderwijs. Aan de hand van enkele applicaties voor Android telefoons zul je de principes van programmeren leren.

\section*{Gebruikte notaties}
\addcontentsline{toc}{section}{Gebruikte notaties}
In dit boek worden enkele notaties gebruikt om het voor jou als lezer makkelijker leesbaar te maken. In deze sectie geven we hier een opsomming van.

\begin{description}
   \item[\block{Block}] Blocks zul je gebruiken om te programmeren. Blocks in \ai worden weergegeven als puzzelstukjes. 
   \item[\menuitem{Knop} of \menuitem{menu}] Menuopties of knoppen worden op deze manier aangegeven.
   \item[\emph{Term}] Belangrijke termen worden schuingedrukt om deze extra te benadrukken.
   \item[\lefthand\ Hints] Naast de tekst worden soms belangrijke hints of tips gegeven.
   \item[Opgaven] Opgaven worden aangeduid met een blauwe balk. 
     \begin{opgave}
       \opgVraag
	Hier staat de vraag.
       \opgUitwerking
         In sommige gevallen wordt ook een uitwerking gegeven.
     \end{opgave}
   \item[Waarschuwing] In sommige gevallen willen we je een belangrijke waarschuwing of tip geven. Deze worden voorzien van een rode balk.
     \begin{derivation}{Waarschuwing}
       Hier staat een waarschuwing, lees deze zorgvuldig!
      \end{derivation}
    \item[Uitproberen!] \runOpTelefoon{} Regelmatig kom je de afbeelding hiernaast tegen. Dit betekent dat je de app waar je op dat moment mee bezig bent uit mag proberen op je telefoon of in de emulator.
\end{description}