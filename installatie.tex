\chapter{Installatie}

De \ai ontwikkelomgeving draait in een webbrowser. Om toegang tot \ai te krijgen heb je een Google account nodig. Je kunt zo'n account aanvragen op: \url{http://accounts.google.com/NewAccount?hl=nl}. Als je een Google account (aangemaakt) hebt kun je inloggen via de volgende link: \url{http://beta.appinventor.mit.edu}.
Nadat je ingelogd bent zie je een scherm zoals afgebeeld in figuur \ref{screenshots/projects}.

\inlinefig{screenshots/projects}{`My Projects' scherm}
 
Op school is alle software die \ai nodig heeft al voor je ge\"installeerd. Maar als je thuis aan de slag gaat is dat misschien niet zo. Daarom volgen hierna de instructies om thuis te zorgen dat de \ai ontwikkelomgeving goed werkt. Als je al een werkende omgeving hebt kun je meteen door naar het volgende hoofdstuk. 

Het is nu belangrijk dat je de software installeert die \ai nodig heeft om de \emph{Blocks Editor} en de emulator uit te voeren. Deze termen zullen je nu misschien nog niets zeggen en dat is op dit moment niet erg. Verderop in het materiaal, als je ze nodig hebt, zullen ze in detail behandeld worden. Het is nu belangrijk om op de `Learn' link te klikken, zie figuur \ref{screenshots/projects_learn}.

\inlinefig{screenshots/projects_learn}{Locatie van de `Learn' link}

Als we dit hebben gedaan zien we het scherm zoals afgebeeld in figuur \ref{screenshots/setup}. Zoals je kunt zien zijn we op dit moment ge\"interesseerd in de `Setup' link. Via deze link komen we namelijk te weten wat we allemaal moeten installeren om te kunnen werken met alle functionaliteit die \ai ons te bieden heeft.
We moedigen je echter ook aan om op de andere links te klikken en even rond te neuzen wat je daar kunt vinden. Sommige links zijn handig als je later nog eens iets op wilt zoeken.

\inlinefig{screenshots/setup}{Locatie van de `Setup' link}

Nadat je op de `Setup' link hebt geklikt, zie figuur \ref{screenshots/setup}, kom je op een pagina met installatie instructies. Voorlopig hoef je alleen `Step 1' te doen `Set up your computer', zie figuur \ref{screenshots/setup_link}.
De tweede stap mag je overslaan omdat we die in dit lesmateriaal met je gaan doorlopen.

\inlinefig{screenshots/setup_link}{Locatie van de `Set up your computer' link}
