\chapter{Installatie}

De \ai ontwikkelomgeving draait in een webbrowser. Om toegang tot \ai te krijgen heb je een Google account nodig. Je kunt zo'n account aanvragen op: \url{http://accounts.google.com/NewAccount?hl=nl} . Als je een Google account (aangemaakt) hebt kun je inloggen via de volgende link: \url{http://appinventor.mit.edu/explore/} en door dan te klikken op \emph{`Create Apps!'} rechtsboven in de hoek.

%\inlinefig{screenshots/createapps}{Create Apps}

Nadat je ingelogd bent en een paar welkomstschermen weggeklikt hebt zie je een scherm zoals afgebeeld in figuur \ref{screenshots/projects}.

\inlinefig{screenshots/projects}{Het `My Projects' scherm} 

Klik op \emph{`Start new project'} en je begint met een (nu nog) lege app. 

Je kunt je eigen app op je \emph{Android smartphone} opstarten. Hiervoor biedt \ai de optie om een zogenaamde \emph{QR-code} op je scherm te laten zien die je met je smartphone met behulp van de \emph{MIT AI2 Companion} app kunt inscannen. Dit is doorgaans de makkelijkste manier en daarom leggen we die hier uit. 

\section{MIT AI2 Companion app}
Om deze app te installeren zoek je met je smartphone in de \emph{Android app store} met de \emph{Play Store} app op \emph{`mit ai2 companion'} en installeert deze app. Wil je testen of dat werkt? In je internet browser klik je in het \emph{`Build'}-menu op \emph{`App ( provide QR-code for .apk )'}.
 
\marginfig{screenshots/install_build_qr}{Het `Build' menu}

App inventor gaat nu alle informatie bij elkaar zoeken van je app die naar je smartphone overgezet moet worden en laat dan een \emph{QR-code} zien die je met de zojuist ge\"installeerde \emph{MIT ai2 companion app} kunt inscannen: 

\qrcode[]{Nee, niet de QR-code in de tekst inscannen: je moet de QR-code op je scherm inscannen!}

de smartphone weet dan het internetadres waar de door jou gebouwde app gedownload kan worden. Als je dit zojuist hebt gedaan met het lege project vraagt je smartphone of je de app wil installeren. Door op \emph{install} te drukken wordt de app ge\"installeerd, en \emph{open} start de app daarna op. Bij het lege project zul je een leeg scherm zien met bovenaan de tekst \emph{`Screen1'}, wat op dit moment wil zeggen: het is je gelukt! Dit werkt alleen als de smartphone toegang heeft tot internet. 

\section{Moet er ook iets ge\"installeerd worden op je computer?}
\marginfig{screenshots/install_connect_menu}{Connect-menu}
Als de hierboven beschreven methode werkt hoef je in eerste instantie verder niets te installeren. Als het niet werkt kun je de \emph{`USB'}-optie proberen. Op den duur kan het prettig zijn op je computer een zogenaamde \emph{emulator} te installeren: deze maakt het mogelijk op de computer je app uit te proberen. 


%Nadat je op de `Setup' link hebt geklikt, zie figuur \ref{screenshots/setup}, kom je op een pagina met installatie instructies. Voorlopig hoef je alleen `Step 1' te doen `Set up your computer', zie figuur \ref{screenshots/setup_link}.
%De tweede stap mag je overslaan omdat we die in dit lesmateriaal met je gaan doorlopen.

%\inlinefig{screenshots/setup_link}{Locatie van de `Set up your computer' link}
